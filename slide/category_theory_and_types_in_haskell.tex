\documentclass[pdflatex,ja=standard]{beamer}
\usepackage{amsmath, amssymb, amsthm, amscd}
\usepackage[all]{xy}

\usepackage[whole]{bxcjkjatype}

\usepackage{beamerthemeshadow}
\usefonttheme{professionalfonts}

% \newcommand{\term}[1][t]{\mathsf{#1}}
% \newcommand{\type}[1][T]{\mathsf{#1}}
% \newcommand{\var}[1][x]{\mathsf{#1}}
% \newcommand{\letbind}[3]{\mathsf{let}\ #1\mathsf{=}#2\ \mathsf{in}\ #3}

\title{圏とHaskellの型}
\author{Kinebuchi Tomohiko}
\date{\today}

\begin{document}
\frame{\titlepage}

%% メモより
% ・下準備
% ・・[プ] 圏論とは
% ・・・ref. Awodey
% ・・・TODO: Awodey読んでおく
% ・・[数] プログラムの型とは
% ・・・型の嬉しさ ref. TAPL
% ・主題
% ・・[プ] [数] 型、関数とHask圏の対応
% ・・・[プ] 具体例: double
% ・・函手
% ・・・[プ] HaskellのFunctor
% ・・・[数] 圏の函手
% ・・・[プ] [数] 両者の類似、Functor則と函手の公理
% ・・モノイダル圏
% ・・・[プ] モノイドとは
% ・・・[プ] モノイドの例、文字列、リスト
% ・・・[数] モノイドの定義
% ・・・[プ] [数] モノイダル圏の定義
% ・・モノイダル函手
% ・・Applicative函手


% 話す内容

\frame{
  \begin{itemize}
  \item 下準備
    \begin{itemize}
    \item スライドの表記について
    \item 圏論とは
    \item プログラムの型とは
    \end{itemize}
  \item Haskellの型を圏として見る
    \begin{itemize}
    \item Hask圏
    \item Functor
    \item Applicative Functor
    \end{itemize}
  \item 参考文献
  \item 宣伝
    \begin{itemize}
    \item 数学茶屋
    \end{itemize}
  \end{itemize}
}

% 対象

\frame{

}

% [プ]

% [数]



\end{document}
