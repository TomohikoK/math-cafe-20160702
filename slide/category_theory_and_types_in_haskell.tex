\documentclass[pdflatex,ja=standard]{beamer}
\usepackage{amsmath, amssymb, amsthm, amscd}
\usepackage[all]{xy}

\usepackage[whole]{bxcjkjatype}

\usepackage{beamerthemeshadow}
\usefonttheme{professionalfonts}

\usepackage{verbatim}

\newcommand{\code}[1]{\texttt{#1}}

\newcommand{\mathint}[0]{\mathbb{Z}}

% \newcommand{\term}[1][t]{\mathsf{#1}}
% \newcommand{\type}[1][T]{\mathsf{#1}}
% \newcommand{\var}[1][x]{\mathsf{#1}}
% \newcommand{\letbind}[3]{\mathsf{let}\ #1\mathsf{=}#2\ \mathsf{in}\ #3}

\title{圏とHaskellの型}
\author{Kinebuchi Tomohiko}
\date{\today}

\begin{document}
\frame{\titlepage}

%% メモより
% ・下準備
% ・・[プ] 圏論とは
% ・・・ref. Awodey
% ・・・TODO: Awodey読んでおく
% ・・[数] プログラムの型とは
% ・・・型の嬉しさ ref. TAPL
% ・主題
% ・・[プ] [数] 型、関数とHask圏の対応
% ・・・[プ] 具体例: double
% ・・函手
% ・・・[プ] HaskellのFunctor
% ・・・[数] 圏の函手
% ・・・[プ] [数] 両者の類似、Functor則と函手の公理
% ・・モノイダル圏
% ・・・[プ] モノイドとは
% ・・・[プ] モノイドの例、文字列、リスト
% ・・・[数] モノイドの定義
% ・・・[プ] [数] モノイダル圏の定義
% ・・モノイダル函手
% ・・Applicative函手


% 話す内容

\frame{
  \frametitle{目次}
  \begin{itemize}
  \item 想定聴者
  \item 下準備
    \begin{itemize}
    \item スライドの表記について
    \item 圏論とは
    \item プログラムの型とは
    \end{itemize}
  \item Haskellの型を圏として見る
    \begin{itemize}
    \item Hask圏
    \item Functor
    \item Applicative Functor
    \end{itemize}
  \item 参考文献
  \item 宣伝
    \begin{itemize}
    \item 数学茶屋
    \end{itemize}
  \end{itemize}
}

% 想定聴者

\frame{
  \frametitle{想定聴者}
  \begin{itemize}
  \item プログラムの型は分かるが圏論を知らない人
  \item 圏論は知っているがプログラム (の型) を知らない人
  \end{itemize}
  → プログラムの型が圏論を通してどう見えるかを解説していきます.
}

\frame{
  \frametitle{下準備 - スライドの表記について}
  スライドのタイトルに目印を付け,
  どちらの想定聴者に向けたスライドなのかを表します.
  \begin{itemize}
  \item プログラムの型は分かるが圏論を知らない人 \\
    → [プ]
  \item 圏論は知っているがプログラム (の型) を知らない人 \\
    → [圏]
  \end{itemize}
}

% Hask圏

\frame{
  \frametitle{[プ][圏] Haskellの型がなす圏 (Hask圏)}
  \begin{itemize}
  \item 対象: 型
  \item 射: 一変数関数
    \begin{itemize}
    \item 恒等射: \code{id}関数 (引数をそのまま返す関数)
    \item 射の合成: 関数の合成
    \end{itemize}
  \end{itemize}
  関数$f$の変数の型が$A$, 返り値の型が$B$のとき, $f$は$A$から$B$への射となる.
}

\frame{
  \frametitle{[プ][圏] 例. Hask圏の対象}
  \begin{itemize}
  \item Integer: 整数の型
  \end{itemize}
}

\frame{
  \frametitle{[プ][圏] 例. Hask圏の射}
  \begin{itemize}
  \item Haskellの関数\code{double}
    \begin{itemize}
    \item \verbatiminput{../code/double.hs}
    \end{itemize}
  \item 数学の関数$double$
    \begin{itemize}
    \item $double: \mathint \to \mathint$ \\
      $double(x) = 2x$
    \end{itemize}
  \end{itemize}
}

% Functor

%% 図式を図式にうつすもの (星座で書く?)

\frame{
  \frametitle{}
  \begin{itemize}
  \item
  \end{itemize}
}


% Applicative Functor

\frame{
  \frametitle{}
  \begin{itemize}
  \item
  \end{itemize}
}


% 参考文献

\frame{
  \frametitle{参考文献}
  
}

% おしまい

\frame{
  \frametitle{[プ][圏] おしまい}
  % 圏と型の対応を再掲
  Hask圏 (再掲)
}

\end{document}
