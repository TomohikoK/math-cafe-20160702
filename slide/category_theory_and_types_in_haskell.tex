\documentclass[pdflatex,ja=standard]{beamer}
\usepackage{amsmath, amssymb, amsthm, amscd}
\usepackage[all]{xy}

\usepackage[whole]{bxcjkjatype}

\usepackage{beamerthemeshadow}
\usefonttheme{professionalfonts}

\usepackage{inconsolata}
\usepackage{verbatim}
\usepackage{hyperref}

\newcommand{\code}[1]{\texttt{#1}}

\newcommand{\mathint}[0]{\mathbb{Z}}

\title{圏とHaskellの型}
\author{Kinebuchi Tomohiko}
\date{\today}

\begin{document}
\frame{\titlepage}

% ・下準備
% ・・[プ] 圏論とは
% ・・・ref. Awodey
% ・・・TODO: Awodey読んでおく
% ・・[数] プログラムの型とは
% ・・・型の嬉しさ ref. TAPL
% ・主題
% ・・[プ] [数] 型、関数とHask圏の対応
% ・・・[プ] 具体例: double
% ・・函手
% ・・・[プ] HaskellのFunctor
% ・・・[数] 圏の函手
% ・・・[プ] [数] 両者の類似、Functor則と函手の公理

%% 全体として[プ]→[圏]の流れ
% 細かく言うと
% ・[プ] 定義 (実コード)
% ・・使い方の例
% ・・意味
% ・[圏] その裏にある概念, その定義
% ・・補足
% ・・プの人向けの補足

% 話す内容

\frame{
  \frametitle{目次}
  \begin{itemize}
  \item 想定聴者
  \item 下準備
    \begin{itemize}
    \item スライドの表記について
    \item 圏論とは
    \item プログラムの型とは
    \end{itemize}
  \item Haskellの型を圏として見る
    \begin{itemize}
    \item Hask圏
    \item Functor
    \item Applicative Functor
    \end{itemize}
  \item 参考文献
  \item 宣伝
    \begin{itemize}
    \item 数学茶屋
    \end{itemize}
  \end{itemize}
}

% 想定聴者

\frame{
  \frametitle{想定聴者}
  \begin{itemize}
  \item プログラムの型は分かるが圏論を知らない人
  \item 圏論は知っているがプログラム (の型) を知らない人
  \end{itemize}
  → プログラムの型が圏論を通してどう見えるかを解説していきます.
}

\frame{
  \frametitle{話すこと}
  \begin{itemize}
  \item Haskellの型の仕組み \\
    % undefined はややこしいので無視
    型全体がある圏と見えること \\
    \begin{tabular}{l|l}
      \hline
      Haskellの型システム & 圏論の概念 \\
      \hline \hline
      型 & 対象 \\
      関数 & 射 \\
      tuple & 直積 \\
      either & 直和 \\
      Functor & 関手 \\
      \hline
    \end{tabular}
  \end{itemize}
}

\frame{
  \frametitle{話さないこと}
  \begin{itemize}
  \item 圏論一般の説明 \\
    前の発表を参照してください
  \item プログラムの圏論的意味論 %[TODO] 表現を確認
  \item Haskellの文法
  \item Haskellの良いプログラミング
  \end{itemize}
}

\frame{
  \frametitle{下準備 - スライドの表記について}
  スライドのタイトルに目印を付け,
  どちらの想定聴者に向けたスライドなのかを表します.
  \begin{itemize}
  \item プログラムの型は分かるが圏論を知らない人 \\
    → [プ]
  \item 圏論は知っているがプログラム (の型) を知らない人 \\
    → [圏]
  \end{itemize}
}

\frame{
  \frametitle{[プ] 圏論とは}
  (あくまでこの発表での圏論の使い方ですが)
  「対象」と対象どうしをつなぐ「射」やそれらの「性質」という言葉で物事を記述する手法.

  対象について語るとき, 他の対象との関係の性質という外部からの視点だけを使い,
  「何からできているか?」という内部からの視点を使わないという特徴がある.

  %[話] API (Javaのインターフェース (or Javadoc?), C, C++のヘッダファイルとか) だけを与えられて実装するようなもの
  %[話] その代わりAPIの事前条件と事後条件についてはきっちり書いてある
}

\frame{
  \frametitle{[プ] 圏論のふわっとした説明??}
  \begin{itemize}
  \item
  \end{itemize}
}

\frame{
  \frametitle{[圏] プログラムの型とは}
  「型」= プログラムに登場する値や式の種類のこと
  % ここでスライドを分けるかも
  型の役割
  \begin{itemize}
  \item ソースコードを読みやすくする \\
    ※ ソースコード = プログラムを記述したもの. テキスト形式のことが多い.
  \item プログラムに間違いが無いことを (ある程度) 保証する
    整数が期待されているところに文字列が現れるような間違いを事前に防ぐ
    % 「事前」の説明を次のスライドで
  \item プログラムの最適化に使われる
    %[TODO] TAPLを調べる
  \end{itemize}
}

\frame{
  \frametitle{[圏] ソースコードのコンパイルについて}

  「プログラムのソースコード」= 人間が読める形式でプログラムの振る舞いを記述したもの
  % たまにバイナリを読める人もいますが
  コンパイルという処理でコンピュータで実行できる形式に変換されるものもある
  % 静的言語, 動的言語の話をする?
  % 実行ファイルになるまでの過程をざっと書く,「型検査」は明記
  \begin{itemize}
  \item
  \end{itemize}
}

\frame{
  \frametitle{}
  \begin{itemize}
  \item
  \end{itemize}
}

\frame{
  \frametitle{}
  \begin{itemize}
  \item
  \end{itemize}
}

% Hask圏

\frame{
  \frametitle{[プ][圏] Haskellの型がなす圏 (Hask圏)}
  \begin{itemize}
  \item 対象: 型
  \item 射: 一変数関数
    \begin{itemize}
    \item 恒等射: \code{id}関数 (引数をそのまま返す関数)
    \item 射の合成: 関数の合成
    \end{itemize}
  \end{itemize}
  関数$f$の変数の型が$A$, 返り値の型が$B$のとき, $f$は$A$から$B$への射となる.
  \url{https://en.wikibooks.org/wiki/Haskell/Category_theory}
}

\frame{
  \frametitle{[プ][圏] 例. Hask圏の対象}
  \begin{itemize}
  \item Integer: 整数の型
  \item String: 文字列の型
  \item [Integer]: 整数のリストの型
  \item Maybe String: 文字列の値があるか値がないことを表す型
  \end{itemize}
}

\frame{
  \frametitle{[プ][圏] 例. Hask圏の射}
  なんだか見た目が似ている.
  \begin{itemize}
  \item Haskellの関数\code{double} \\
    \verbatiminput{../code/double.hs}
  \item 数学の関数$double$ \\
    $double: \mathint \to \mathint$ \\
    $double(x) = 2x$
  \end{itemize}
}

\frame{
  \frametitle{[プ][圏] 例. Hask圏の射の始域と終域}
  % 関数の型宣言の情報しか使わない
  \begin{itemize}
  \item
  \end{itemize}
}

\frame{
  \frametitle{[プ][圏] Haskellの型が圏をなすことの説明}
  \begin{itemize}
  \item 恒等射の存在: \code{id: a -> a} (\code{a}は任意の型)
  \item 射の合成: Haskellの意味での関数の合成, \code{f . g}
  \end{itemize}
  任意の関数\code{f, g}に対して\code{f . id = f, id . g = g}が成り立つ.
  % 面倒なのでundefinedと副作用は無いものとする
}

%% idのついでに同型な対象の話もする? 例は何だろ? (正直あまり思い浮かばない, Evenを作っておくくらいかな?)

\frame{
  \frametitle{}
  \begin{itemize}
  \item
  \end{itemize}
}

\frame{
  \frametitle{}
  \begin{itemize}
  \item
  \end{itemize}
}


% Functor

%% 図式を図式にうつすもの (星座で書く?)

%% 型クラスの説明どうしよ? → 敢えて説明しない

%% 例はMaybeとListにしよう

\frame{
  \frametitle{[プ] HaskellのFunctor}
  \verbatiminput{../code/functor.hs}
  \url{https://hackage.haskell.org/package/base-4.9.0.0/docs/src/GHC.Base.html\#Functor}
}

\frame{
  \frametitle{[圏] 関手}
  % 解説はされてるので, ここではさらっと思い出す
  \begin{itemize}
  \item
  \end{itemize}
}

\frame{
  \frametitle{[プ] ファンクタ則 (Functor law)}
  HaskellのFunctorが関手になること
  \begin{itemize}
  \item
  \end{itemize}
}

\frame{
  \frametitle{[プ] 例. Maybe (1/2)}
  % Maybe自体の解説, 何を表現しているのか, 元の型の拡張
  \begin{itemize}
  \item
  \end{itemize}
}

\frame{
  \frametitle{[プ] 例. Maybe (2/2)}
  % Maybeが関手になること, fmap showとかで説明
  \begin{itemize}
  \item
  \end{itemize}
}

\frame{
  \frametitle{[プ] 例. List (1/2)}
  % List [] 自体の解説, 何を表現しているのか
  \begin{itemize}
  \item
  \end{itemize}
}

\frame{
  \frametitle{[プ] 例. List (2/2)}
  % Listが関手になること
  % [圏] 関数f(x)を自然にf({x})に拡張するのと同じ
  \begin{itemize}
  \item
  \end{itemize}
}

\frame{
  \frametitle{[圏] 自然変換}
  % 解説はされてるので, ここではさらっと思い出す
  \begin{itemize}
  \item
  \end{itemize}
}

\frame{
  \frametitle{[プ] Hask圏で自然変換に対応するものは?}
  \begin{itemize}
  \item
  \end{itemize}
}

\frame{
  \frametitle{[プ] 自然変換になる関数が持つ性質}
  % たいていは型パラメータの違いは影響しない
  \begin{itemize}
  \item
  \end{itemize}
}

\frame{
  \frametitle{[プ] 例. リストの先頭の要素を取り出す関数}
  % length, takeとか
  \begin{itemize}
  \item
  \end{itemize}
}

\frame{
  \frametitle{}
  \begin{itemize}
  \item
  \end{itemize}
}

\frame{
  \frametitle{}
  \begin{itemize}
  \item
  \end{itemize}
}


% 参考文献

\frame{
  \frametitle{参考文献}
  % H本
  % Hackage
  % WikiBooks https://en.wikibooks.org/wiki/Haskell/Category_theory
}

% 宣伝

\frame{
  \frametitle{数学茶屋}
  \begin{itemize}
  \item
  \end{itemize}
}

\frame{
  \frametitle{グッド・マス?}
  \begin{itemize}
  \item
  \end{itemize}
}



% おしまい

\frame{
  \frametitle{[プ][圏] おしまい}
  % 圏と型の対応を再掲
  Hask圏 (再掲)
  \begin{tabular}{l|l}
    \hline
    Haskellの型システム & 圏論の概念 \\
    \hline \hline
    型 & 対象 \\
    関数 & 射 \\
    tuple & 直積 \\
    either & 直和 \\
    Functor & 関手 \\
    ?? & 自然変換 \\
    \hline
  \end{tabular}
}

\end{document}
