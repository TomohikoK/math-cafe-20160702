\documentclass[pdflatex,ja=standard]{beamer}
\usepackage{amsmath, amssymb, amsthm, amscd}
\usepackage[all]{xy}

\usepackage[whole]{bxcjkjatype}

\usepackage{beamerthemeshadow}
\usefonttheme{professionalfonts}

\usepackage{inconsolata}
\usepackage{verbatim}
\usepackage{hyperref}
\usepackage[all]{xy}
\usepackage{framed}

\newcommand{\code}[1]{\texttt{#1}}

\newcommand{\mathint}[0]{\mathbb{Z}}

\newcommand{\category}[1]{\mathcal{#1}}

\title{圏とHaskellの型}
\author{Kinebuchi Tomohiko}
\date{\today}

\begin{document}
\frame{\titlepage}

% ・下準備
% ・・[プ] 圏論とは
% ・・[数] プログラムの型とは
% ・・・型の嬉しさ ref. TAPL
% ・主題
% ・・[プ] [数] 型、関数とHask圏の対応
% ・・・[プ] 具体例: double
% ・・函手
% ・・・[プ] HaskellのFunctor
% ・・・[数] 圏の函手
% ・・・[プ] [数] 両者の類似、Functor則と函手の公理

%% 全体として[プ]→[圏]の流れ
% 細かく言うと
% ・[プ] 定義 (実コード)
% ・・使い方の例
% ・・意味
% ・[圏] その裏にある概念, その定義
% ・・補足
% ・・プの人向けの補足

% 話す内容

\frame{
  \frametitle{目次}
  \begin{itemize}
  \item 想定聴者
  \item 下準備
    \begin{itemize}
    \item スライドの表記について
    \item 圏論とは
    \item プログラムの型とは
    \end{itemize}
  \item Haskellの型全体を圏として見る
    \begin{itemize}
    \item Hask圏
    \item tuple型
    \item Either型
    \item Functor
    \item 自然変換
    \end{itemize}
  \item 参考文献
  \item 宣伝
    \begin{itemize}
    \item 数学茶屋
    \end{itemize}
  \end{itemize}
}

% 想定聴者

\frame{
  \frametitle{想定聴者}
  \begin{itemize}
  \item プログラムの型は分かるが圏論を知らない人
  \item 圏論は知っているがプログラムの型を知らない人
  \end{itemize}
  → プログラムの型が圏論を通してどう見えるかを解説していきます.
}

\frame{
  \frametitle{話すこと}
  \begin{itemize}
  \item Haskellの型の仕組みと圏論の対応 \\
    \begin{tabular}{l|l}
      \hline
      Haskellの型システム & 圏論の概念 \\
      \hline \hline
      型 & 対象 \\
      関数 & 射 \\
      tuple & 直積 \\
      either & 直和 \\
      Functor & 関手 \\
      (ある種の関数) & 自然変換 \\
      \hline
    \end{tabular}
  \end{itemize}
  Haskellの型システムが圏論に支えられている実例を見ていきます. \\
}

\frame{
  \frametitle{話さないこと}
  \begin{itemize}
  \item 圏論一般の説明 \\
    前の発表を参照してください
  \item プログラムの圏論的意味論
  \item Haskellの文法
  \item Haskellの良いプログラミング
  \end{itemize}
  この発表を聞いてもHaskellが書けるようになるわけではないです. 悪しからず.
}

\frame{
  \frametitle{下準備 - スライドの表記について}
  スライドのタイトルに目印を付け, どちら側の話なのかを表します.
  \begin{itemize}
  \item プログラミング側の話 → [プ]
  \item 圏論側の話 → [圏]
  \end{itemize}
  %[TODO] 圏論知ってる人? プログラミング知ってる人? 型と圏の話を知ってる人? と質問して手を挙げてもらう
  %[TODO] GitHubのURLをどこかに書く
}

\frame{
  \frametitle{スライドの入手先}
  \url{https://github.com/TomohikoK/math-cafe-20160702}
  LaTeXファイルを手に入れて嬉しい人向け
}

\frame{
  \frametitle{[圏] 圏論とは}
  (あくまでこの発表での圏論の使い方ですが)

  「対象」と対象どうしをつなぐ「射」やそれらの「性質」という言葉で物事を記述する手法.

  対象について語るとき, 他の対象との関係の性質という外部からの視点だけを使い,
  「何からできているか?」という内部からの視点を使わないという特徴がある.

  %[話] API (Javaのインターフェース (or Javadoc?), C, C++のヘッダファイルとか) だけを与えられて実装するようなもの
  %[話] その代わりAPIの事前条件と事後条件についてはきっちり書いてある
}

\frame{
  \frametitle{[プ] プログラムの型とは}
  「型」とはプログラムに登場する値や式の種類のことで, その役割は
  \begin{itemize}
  \item プログラムの間違いの検出 \\
    整数が期待されているところに文字列が現れるなどの間違いを事前に見付ける.
    % 「事前」の説明を次のスライドで
  \item 抽象化, インターフェースの定義 \\
    値や関数の重要な性質だけに着目する.
  \item ソースコードを読みやすくする. \\
    「ソースコード」とはプログラムを記述したもの.
    % 「ソースコード」の説明を次のスライドで
    % 言語の安全性については説明し切れないので割愛
  \item プログラムの最適化 \\
    型の情報を使って, より高速なプログラムを生成する.
  \end{itemize}
  ref.『Types and Programming Languages』(Benjamin C. Pierce)
}

% Hask圏

\frame{
  \frametitle{[プ][圏] Haskellの型がなす圏 (Hask圏)}
  \begin{itemize}
  \item 対象: 型
  \item 射: 一変数関数
    \begin{itemize}
    \item 恒等射: $\code{id}$関数 (引数をそのまま返す関数)
    \item 射の合成: $\code{f . g}$ (Haskellの意味での関数の合成)
    \end{itemize}
  \end{itemize}
  関数$f$の変数の型が$A$, 返り値の型が$B$のとき, $f$を$A$から$B$への射とします.
  (\url{https://en.wikibooks.org/wiki/Haskell/Category_theory})
  %[話] \code{undefined}や副作用とかややこしいので純粋な関数だけにしておく
}

\frame{
  \frametitle{[プ][圏] Haskellの型が圏をなすことの説明}
  \begin{itemize}
  \item 恒等射についての条件:
    \begin{itemize}
    \item 任意の関数$\code{f}, \code{g}$に対して \\
      $\code{f . id == f}$と$\code{id . g == g}$が成り立ちます.
    \end{itemize}
  \item 射の合成についての条件:
    \begin{itemize}
    \item 任意の関数$\code{f}, \code{g}, \code{h}$に対して \\
      $\code{(h . g) . f == h . (g . f)}$が成り立ちます.
    \end{itemize}
  \end{itemize}
  → Haskellの型が圏をなすことが分かりました.
  % 面倒なのでundefinedと副作用は無いものとする
}

\frame{
  \frametitle{[プ][圏] 例. Hask圏の対象}
  % 対象が値ではないことに注意
  \begin{itemize}
  \item \code{Integer}: 整数の型
  \item \code{String}: 文字列の型
  \item \code{Maybe String}: 文字列の値がある状態もしくは値が無い状態を表す型
  \item \code{[Integer]}: 整数のリストの型
  \end{itemize}
}

\frame{
  \frametitle{[プ][圏] 例. Hask圏の射}
  Haskellの世界の$\code{double}$と数学の世界の$double$の見た目はなんだか似ている.
  \begin{itemize}
  \item Haskellの関数\code{double} \\
    \verbatiminput{../code/double.hs}
  \item 数学の関数$double$ \\
    $double: \mathint \to \mathint$ \\
    $double(x) = 2x$
  \end{itemize}

  $\code{double}$関数を射と見たときの始域は$\code{Integer}$で終域も$\code{Integer}$です.
  $\code{double}$関数の型宣言の情報のみを使っていて, 関数の定義の情報は使っていないことに注意してください.
}

\frame{
  \frametitle{[プ][圏] ここまでの内容で質疑応答}
  \begin{itemize}
  \item 「あのスライドをもう1回見せてほしい」
  \item 「あそこの解説を聞き逃した」
  \item 「あれってどういう意味?」
  \end{itemize}
}

\frame{
  \frametitle{[プ] Haskellの直積型 (tuple型)}
  型\code{a}, \code{b}に対し, それらの直積と呼ばれるtuple型\code{(a, b)}が存在し,
  tupleの基本的な関数として\code{fst}, \code{snd}があります.
  \begin{framed}
    \verbatiminput{../code/tuple.hs}
  \end{framed}
  (\url{https://hackage.haskell.org/package/base-4.9.0.0/docs/src/Data.Tuple.html\#line-33})

  → これに対応する圏論の概念は何でしょう?
}

\frame{
  \frametitle{[圏] 積の復習}
  % 一般の圏の図式
  圏$\category{C}$の対象$X$, $Y$の積$\langle X \times Y, p, q \rangle$とは,
  圏$\category{C}$の対象$X \times Y$,
  射$p: X \times Y \rightarrow X$,
  射$q: X \times Y \rightarrow Y$の組で, 次の条件を満たすものです.
  \def\objectstyle{\displaystyle}
  \def\labelstyle{\displaystyle}
  \centerline{
    \xymatrix@=60pt{
      & W \ar[ld]_{k} \ar@{.>}[d]_{{}^{\exists1}h} \ar[rd]^{l} & \\
      X & X \times Y \ar[l]^{p} \ar[r]_{q} & Y
    }
  }
  (条件)
  任意の圏$\category{C}$の対象$W$と射$k: W \rightarrow X$, $l: W \rightarrow Y$に対して,
  射$h: W \rightarrow X \times Y$が1つだけあって
  \begin{align*}
    p \circ h &= k \\
    q \circ h &= l    
  \end{align*}
  となる.
}

\frame{
  \frametitle{[プ][圏] 直積型がHask圏の積であること}
  % Hask圏の図式
  直積型がHask圏の積になるためには次の条件を満たす必要があります. \\
  (条件) $\code{a}$, $\code{b}$, $\code{c}$を任意の型とし,
  $\code{k :: c -> a}$, $\code{l :: c -> b}$を任意の関数としたとき,
  関数$\code{h :: c -> (a, b)}$があって
  \begin{align*}
    \code{fst . h == k} \\
    \code{snd . h == l}
  \end{align*}
  とできる.
  \def\objectstyle{\displaystyle}
  \def\labelstyle{\displaystyle}
  \centerline{
    \xymatrix@=60pt{
      & \code{c} \ar[ld]_{\code{k}} \ar@{.>}[d]_{{}^{\exists1}\code{h}} \ar[rd]^{\code{l}} & \\
      \code{a} & \code{(a, b)} \ar[l]^{\code{fst}} \ar[r]_{\code{snd}} & \code{b}
    }
  }
  → $\code{h x = (k x, l x)}$と実装すれば条件を満たします.
}

\frame{
  \frametitle{[プ] $\code{h}$の例}
  作為的な例ですが.
  \begin{framed}
    \verbatiminput{../code/tuple_h.hs}
  \end{framed}
}

\frame{
  \frametitle{[プ] Haskellの直和型 (Either型)}
  型\code{a}, \code{b}に対し, それらの直和と呼ばれるEither型\code{Either a b}が存在し,
  Eitherのコンストラクタとして\code{Left}, \code{Right}があります.
  \begin{framed}
    \verbatiminput{../code/either.hs}
  \end{framed}
  (\url{https://hackage.haskell.org/package/base-4.9.0.0/docs/src/Data.Either.html\#Either})

  → これに対応する圏論の概念は何でしょう?
}

\frame{
  \frametitle{[圏] 余積の復習}
  % 一般の圏の図式
  圏$\category{C}$の対象$X$, $Y$の余積$\langle X \coprod Y, i, j \rangle$とは,
  圏$\category{C}$の対象$X \coprod Y$,
  射$i: X \rightarrow X \coprod Y$,
  射$j: Y \rightarrow Y \coprod Y$の組で, 次の条件を満たすものです.
  \def\objectstyle{\displaystyle}
  \def\labelstyle{\displaystyle}
  \centerline{
    \xymatrix@=60pt{
      & W & \\
      X \ar[ru]^{k} \ar[r]_{i} & X \coprod Y \ar@{.>}[u]_{{}^{\exists1}h} & Y \ar[l]^{j} \ar[lu]_{l}
    }
  }
  (条件)
  任意の圏$\category{C}$の対象$W$と射$k: X \rightarrow W$, $l: Y \rightarrow W$に対して,
  射$h: X \coprod Y \rightarrow W$が1つだけあって
  \begin{align*}
    h \circ i &= k \\
    h \circ j &= l    
  \end{align*}
  となる.
}

\frame{
  \frametitle{[プ][圏] 直和型がHask圏の余積であること}
  % Hask圏の図式
  直和型がHask圏の余積になるためには次の条件を満たす必要があります. \\
  (条件) $\code{a}$, $\code{b}$, $\code{c}$を任意の型とし,
  $\code{k :: a -> c}$, $\code{l :: b -> c}$を任意の関数としたとき,
  関数$\code{h :: Either a b -> c}$があって
  \begin{align*}
    \code{h . Left == k} \\
    \code{h . Right == l}
  \end{align*}
  とできる.

  \def\objectstyle{\displaystyle}
  \def\labelstyle{\displaystyle}
  \centerline{
    \xymatrix@=60pt{
      & \code{c} & \\
      \code{a} \ar[ru]^{k} \ar[r]_{i} & \code{Either a b} \ar@{.>}[u]_{{}^{\exists1}h} & \code{b} \ar[l]^{j} \ar[lu]_{l}
    }
  }
}

\frame{
  \frametitle{[プ] $\code{h}$の例}
  \begin{framed}
    \verbatiminput{../code/either_h.hs}
  \end{framed}
  Haskellのパターンマッチで$\code{h}$を定義しているので,
  条件を満たすことが分かりやすくなっています.
}

\frame{
  \frametitle{[プ][圏] ここまでの内容で質疑応答}
  \begin{itemize}
  \item 「あのスライドをもう1回見せてほしい」
  \item 「あそこの解説を聞き逃した」
  \item 「あれってどういう意味?」
  \end{itemize}
}


% Functor

%% 型クラスの説明どうするか? → 敢えて説明しない

%% 例はMaybeとList

% ↓ 並びを見直す

\frame{
  \frametitle{[プ] HaskellのFunctor}
  定義は以下の通りです.
  \begin{framed}
    \verbatiminput{../code/functor.hs}
  \end{framed}
  (\url{https://hackage.haskell.org/package/base-4.9.0.0/docs/src/GHC.Base.html\#Functor})

  ある具体的なFunctorを$\code{f}$として, $\code{f}$の射関数$\code{fmap}$の型を宣言しています. \\
  (「射関数」= 関手$F$によって射$f$を射$Ff$に写す関数のこと)
}

\frame{
  \frametitle{[圏] 関手の復習}
  % 数学の関手の定義再掲
  圏$\category{C}$から圏$\category{D}$への関手$F: \category{C} \rightarrow \category{D}$とは,
  対応の組$F: Ob(\category{C}) \rightarrow Ob(\category{D})$,
  $F: \mathtt{Hom}_{\category{C}}(X, Y) \rightarrow \mathtt{Hom}_{\category{D}}(FX, FY)$であって,
  \begin{align*}
    F1_{X} &= 1_{FX} \\
    F(g \circ f) &= Fg \circ Ff
  \end{align*}
  を満たすもの.
}

\frame{
  \frametitle{[プ] HaskellのFunctor則 (Functor law)}
  % Functor則と関手の公理に対比
  Functor則とはFunctorが満たすべき条件.
  これに反する実装自体はできるものの, そうするメリットはあまり無いです.
  \begin{framed}
    \verbatiminput{../code/functor_law.hs}
  \end{framed}
  (\url{https://hackage.haskell.org/package/base-4.9.0.0/docs/Data-Functor.html\#t:Functor})

  このFunctor則の2つの条件が, 関手が満たすべき2つの条件に対応しています. \\
  → HaskellのFunctorと圏論での関手に対応
}

% 例. Maybe

\frame{
  \frametitle{[プ] $\code{Functor}$の例: $\code{Maybe}$}
  % Maybe自体の解説, 何を表現しているのか, 元の型の拡張
  $\code{Maybe}$とは, ある型の値がある状態と値がない空の状態をいっぺんに表せる型です.
  定義は以下の通りです.
  \begin{framed}
    \verbatiminput{../code/maybe.hs}
  \end{framed}
  (\url{https://hackage.haskell.org/package/base-4.9.0.0/docs/src/GHC.Base.html\#Maybe})

  $\code{Nothing}$が「値が無いこと」に対応し, \\
  $\code{Just a}$が「$\code{a}$という型の値があること」に対応しています.
}

\frame{
  \frametitle{[プ] $\code{Maybe}$がFunctor則を満たすこと (1/2)}
  $\code{Maybe}$の$\code{Functor}$としての定義は以下の通りです.
  \begin{framed}
    \verbatiminput{../code/functor_maybe.hs}
  \end{framed}
  (\url{https://hackage.haskell.org/package/base-4.9.0.0/docs/src/GHC.Base.html\#line-652})

  $\code{Maybe a}$について$\code{Nothing}$の場合と$\code{Just a}$の場合のそれぞれで
  Functor則を満たすことを確認すれば良いです.
}

\frame{
  \frametitle{[プ] $\code{Maybe}$がFunctor則を満たすこと (2/2)}
  \begin{framed}
    \verbatiminput{../code/proof_maybe.hs}
  \end{framed}
}

\frame{
  \frametitle{[プ][圏] \code{Maybe}で関数を写してみよう}
  % Maybeが関手になること, fmap showとかで説明

  例: 整数の値を取って, その文字列表現を返す関数$\code{int2str}$.

  \begin{framed}
    \verbatiminput{../code/int2str.hs}    
  \end{framed}

  この関数を$\code{Maybe}$で写すと,
  \begin{align*}
    \code{fmap int2str :: Maybe Integer -> Maybe String}
  \end{align*}
  という関数になります.
  \xymatrix{
    \text{Hask圏} & & \text{Hask圏} & \\
    \code{Integer} \ar[r]^{\code{int2str}} & \code{String} \ar@{=>}[r]^-{\code{Maybe}} & \code{Maybe Integer} \ar[r]^{\code{fmap int2str}} & \code{Maybe String}
    \save "1,1"."1,2"."2,1"."2,2"*\frm<10pt>{-}
    \restore
    \save "1,3"."1,4"."2,3"."2,4"*\frm<10pt>{-}
    \restore
  }
}

\frame{
  \frametitle{[プ] ghciで動かしてみる}
  \begin{framed}
    \verbatiminput{../code/maybe_int2str.ghci}
  \end{framed}
}

% 例. List

\frame{
  \frametitle{[プ] $\code{Functor}$例. List}
  % List [] 自体の解説, 何を表現しているのか
  % [圏] 関数f(x)を自然にf({x})に拡張するのと似ている
  Listとは, ある型の値を0個以上並べたものの型です.
  定義は以下の通りです
  \begin{framed}
    \verbatiminput{../code/list.hs}
  \end{framed}
  (\url{https://hackage.haskell.org/package/ghc-prim-0.4.0.0/candidate/docs/src/GHC-Types.html\#line-33})

  右辺の$\code{[]}$が空のリストを表し, \\
  $\code{a : [a]}$がリストの前に要素を1つ追加したものを表します. \\
  リストの定義では再帰的な定義が使われています.
}

\frame{
  \frametitle{[プ] ListがFunctor則を満たすこと (1/3)}
  Listの$\code{Functor}$としての定義は以下の通りです.
  \begin{framed}
    \verbatiminput{../code/functor_list.hs}
  \end{framed}
  (\url{https://hackage.haskell.org/package/base-4.9.0.0/docs/src/GHC.Base.html\#line-732})
}

\frame{
  \frametitle{[プ] ListがFunctor則を満たすこと (2/3)}
  $\code{map}$関数の定義は以下の通りです.
  \begin{framed}
    \verbatiminput{../code/list_map.hs}
  \end{framed}
  (\url{https://hackage.haskell.org/package/base-4.9.0.0/docs/src/GHC.Base.html\#line-868})

  Listについて$\code{[]}$の場合と$\code{a : [a]}$の場合のそれぞれで
  Functor則を満たすことを確認すれば良いです.
}

\frame{
  \frametitle{[プ] ListがFunctor則を満たすこと (3/3)}
  \begin{framed}
    \verbatiminput{../code/proof_list.hs}
  \end{framed}
}

\frame{
  \frametitle{[プ][圏] Listで関数を写してみよう}
  % Listが関手になること, fmap showとかで説明

  例: 整数の値を取って, その文字列表現を返す関数$\code{int2str}$.

  \begin{framed}
    \verbatiminput{../code/int2str.hs}    
  \end{framed}

  この関数をListで写すと,
  \begin{align*}
    \code{fmap int2str :: [Integer] -> [String]}
  \end{align*}
  という関数になります.
  \xymatrix{
    \text{Hask圏} & & \text{Hask圏} & \\
    \code{Integer} \ar[r]^{\code{int2str}} & \code{String} \ar@{=>}[r]^-{List} & \code{[Integer]} \ar[r]^{\code{fmap int2str}} & \code{[String]}
    \save "1,1"."1,2"."2,1"."2,2"*\frm<10pt>{-}
    \restore
    \save "1,3"."1,4"."2,3"."2,4"*\frm<10pt>{-}
    \restore
  }
}

\frame{
  \frametitle{[プ] ghciで動かしてみる}
  \begin{framed}
    \verbatiminput{../code/list_int2str.ghci}
  \end{framed}
}

\frame{
  \frametitle{[プ][圏] ここまでの内容で質疑応答}
  \begin{itemize}
  \item 「あのスライドをもう1回見せてほしい」
  \item 「あそこの解説を聞き逃した」
  \item 「あれってどういう意味?」
  \end{itemize}
}

% さてここから自然変換

\frame{
  \frametitle{[圏] 自然変換の復習}
  % 解説はされてるので, ここではさらっと思い出す
  圏$\category{C}$, $\category{D}$と関手$F, G: \category{C} \rightarrow \category{D}$があったとき,
  自然変換$\alpha: F \rightarrow G$とは射の集まり$\alpha_{X}: FX \rightarrow GX\ (X \in Ob(\category{C}))$であって,
  $\forall f: X \rightarrow Y$に対し$\alpha_{Y} \circ Ff = Gf \circ \alpha_{X}$を満たすもの
  \def\objectstyle{\displaystyle}
  \def\labelstyle{\displaystyle}
  \begin{framed}
    \centerline{
      \xymatrix@=60pt{
        X \ar[r]^{f} & Y & FX \ar[r]^{Ff} \ar[d]^{\alpha_X} & FY \ar[d]^{\alpha_Y} \\
        & & GX \ar[r]^{Gf} & GY
      }
    }
  \end{framed}
  → さてHask圏で自然変換になるものは何があるでしょう?
}

\frame{
  \frametitle{[プ] 例. リストの先頭の要素を取り出す関数}
  % takeとか, List -> Maybe
  % たいていは型パラメータの違いは影響しない
  \begin{framed}
    \verbatiminput{../code/takeFirst.hs}
  \end{framed}
}


% 参考文献

\frame{
  \frametitle{参考文献}
  \begin{itemize}
  \item 『』
  \end{itemize}
  % H本
  % Hackage
  % WikiBooks https://en.wikibooks.org/wiki/Haskell/Category_theory
  % 圏論の基礎
}

% 宣伝

\frame{
  \frametitle{数学茶屋}
  「フリートーク中心の数学好きの集まり」
  \begin{itemize}
  \item 和コンセプト
  \item 時期は9月
  \item 場所は渋谷近辺
  \item 人数は20人くらい
  \item Facebook → \url{https://www.facebook.com/MathTeaHouse}
  \end{itemize}
}

% おしまい

\frame{
  \frametitle{[プ][圏] おしまい}
  Haskellの型と圏論の概念の対応 (再掲)
  \centerline{
    \begin{tabular}{l|l}
      \hline
      Haskellの型システム & 圏論の概念 \\
      \hline \hline
      型 & 対象 \\
      関数 & 射 \\
      tuple型 & 直積 \\
      Either型 & 直和 \\
      Functor & 関手 \\
      Listのtakeなど & 自然変換 \\
      \hline
    \end{tabular}
  }
}

\end{document}
